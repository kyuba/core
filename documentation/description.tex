\documentclass[a4paper,twoside,titlepage]{article}
\usepackage[top=3cm,bottom=3cm,left=3cm,right=3cm]{geometry}
\usepackage{url}

\renewcommand\familydefault{\sfdefault}

\title{Kyuuba}
\author{Magnus Achim Deininger}

\begin{document}

Kyuuba is an init programme for POSIX-ish systems, that is intended to replace
the current de-facto standard SysV-style and BSD-style init programmes.

Design goals include complete independence of shell scripts, speed, small memory
and on-disk footprints, portability (it should run on at least Linux, *BSD,
Darwin and all applicable parts should run on Windows), flexibility and it
should aid in virtualisation and service jailing.

\paragraph{Note}
When speaking of Windows in this article, this specifically includes
re-implementations such as ReactOS.

\paragraph{Note}
This is the original description paper i wrote before actually starting on kyuba
and it hasn't really been audited much since then, except for converting it from
docbook to latex.

\section*{Introduction}
Kyuuba isn't a completely new programme, actually it's a redesign and rewrite of
eINIT. eINIT was, and still is, modular and fast, but we hit a good deal of
obstacles that can't be fixed due to the original design, and at the same time
due to nobody really enforcing that original design.

A lot of the basic goals for kyuba have thus been influenced by eINIT and the
experience we've gotten with that. Our goals are:

\begin{description}
\item[Speed]
Computers should boot fast, and there is absolutely no reason to have them boot
slow like they do these days. This is especially important for any embedded
application and for things like backup servers that shouldn't be running all the
time, but that need to be able to get online as fast as possible. For these
applications, slow bootup times are generally annoying and often make the
manufacturer of the device look unprofessional.

\item[Small Size (Memory- and On-Disk)]
Init programmes usually need to run at all times, thus they count against the
available RAM at all times. For that reason, an init should be as small as
possible so that the programmes it initialises have as much ram available to
them as possible.

\item[Parallelisation]
A lot of time in the boot process is spent waiting on other processes to do
something. Doing nothing while waiting is generally a bad thing to do if we want
the whole bootup process to be fast, especially considering that some tasks
always take time, such as getting an IP address via DHCP.

\item[Modularity]
Our goal is to get a core application with a 1.0 release out as fast as
possible, but we can't possibly foresee all the potential applications of this
programme beforehand. For this reason, we need a strong module interface, so
that extra functionality can easily added to the core, without modifying the
core.

\item[No Shell Scripting]
Shell scripts are slow, but they're handy. For this reason, shell scripting is
still extremely popular among Unix veterans. But just because shell scripting is
handy, and it's perceived as one way to add flexibility, it isn't the only way
to add flexibility. With a properly designed client library, even C should be
easy enough to use to cover the typical applications of Shell Scripting in the
bootup process.

\item[No 'real' Threading]
The first few versions of eINIT relied on heavy threading, but that turned out
to create a lot of problems. Eventually the overhead of threading made that
approach slower than a custom "green thread" implementation that we tried in a
few unreleased versions.

Green threads as well as message-passing style threads should suffice for our
applications, since there are no calculations that could benefit from
multi-processing, and most of time init is only supposed to spawn a number of
other processes in the right order.

\item[Service Supervision]
We think that supervising running services is exactly the type of thing that an
init programme should be doing. If a service dies for some reason, oftentimes
users will want that service to respawn as fast as possible, or they want some
other actions to be taken, and since init needs to supervise all processes
anyway, this is exactly the type of thing it should be doing at the same time.

\item[Proper handling of unexpected Situations]
If a daemon fails to start, there's often alternative daemons that may take its
place; for example if a user is only interested in a graphical login screen and
using the computer with a GUI, and after updating his gnome installation for
some reason gdm keeps failing, the user will probably prefer a kdm or slim login
screen instead of a text mode prompt, which would leave him with a plain shell
after entering his login data instead of some GUI the user is actually used to.

\item[No need to follow legacy Examples]
One of the things that severely hurt eINIT was that we tried to keep all the
legacy applications working. While this will still be possible with a properly
modular design, we made the mistake of doing this right from the start instead
of just adding it afterwards. This resulted in a lot of confusion with both the
users and developers as to what behaviour can be expected in some circumstances.

\item[Portability]
Even though the init programme is normally considered to be one of the prime
examples of a system-dependant application, we'd like to make sure that this
programme will run on all the usual operating systems: *BSD, Linux, Darwin,
hopefully also Windows (at least some portions of kyuuba should run perfectly
fine on Windows).

\end{description}

Additionally, the programme will be licenced under a BSD-style licence, which
means it will be truly free, but also properly usable in a business setting.

\section{Kyuuba Core Components}
Kyuuba will be split into several components that should interact with each
other. Splitting kyuuba into multiple programmes will help with reclaiming
memory used by components that aren't used very often, and it increases
stability and modularity of the whole system.

\subsection{C Library}
Even though C is perfectly suited for writing programmes that work fast as well
as doing so with tight memory constraints, it's also notorious for needing quite
an effort to keep the code portable. For this reason, we'll be writing a special
C library that serves as a portability layer, as well as a high level
abstraction layer.

The library is written in such a way that it is easily made freestanding, that
is, so that it won't require the host libc, in order to reduce memory use. It's
also designed to only support the functions kyuuba will be needing.

Work on this library has already started under codename `Curie'. The library
includes almost all the features we'll be needing, and the freestanding variants
turned out to be significantly smaller than most of the host libcs we compared
it against. (typically about 16k in statically linked amd64 binaries vs. about
300-600k).

\subsection{The Monitor}
The monitor is kyuuba's equivalent of the regular /sbin/init programme, the
component that is always running as PID=1.

The purpose of this programme is to spawn the coordinators, reap dead processes
as well as keep track of the service statuses as spawned by the coordinators and
keep track of any status information from suspended coordinators and servers.

There's going to be slightly different implementations of this programme,
depending on the host OS, and othe factors. For example, if the monitor is run
off an initramfs as opposed to running directly off a target partition, it will
also need to mount the required root filesystems.

A minimalistic implementation of this would be running directly off a root
filesystem and thus not require any mounting logic or additional configuration.

\subsection{The Coordinators}
The coordinators are daemons spawned by the monitor to coordinate service status
changes and dealing with configuration data. Coordinators are distinctly
different programmes from the monitors, because they may need to need more
fine-grained adjustments to the configuration format of a given distribution
it's supposed to coordinate.

Coordinators will also have the ability to be suspended and resumed. Data needed
for resuming a coordinator will be sent to the launching monitor and provided to
the coordinator by the monitor when resuming.

\subsection{Servers}
Anything that needs to run longer than a coordinator, and which extends the
functionality of kyuuba by doing so, will be called a server. Examples of such
programmes would be daemons that keep track of computer status changes, such as
a closing/opening the lid on laptops, pressing the power or reset button, power
outages with a connected APU, connecting or disconnecting an AC adapter on a
laptop or other battery-powered device or even supplying an emulation for the
/dev/initctl interface of sysv-style init programmes.

Servers will also be used to provide user feedback. This means that neither the
monitors, nor the coordinators will include code to notify the user of any
changes; Instead they will only log any activities, and a specialised server
will then do appropriate feedback, for example it would print startup messages
on a system console, or do logging for the system to syslog. In eINIT, this
rather radical approach proved to be very useful, as it allowed to bring up the
boot messages on a terminal emulator under X, and it allows for alternative
forms of feedback to be developed, such as aural feedback for computers to be
used by blind persons, or for situations where visual feedback is otherwise
inappropriate or even unavailable, such as appliances without any display
devices. With a good TTS system, this could be used to vocalise system warning
messages to the user of a computer or appliance running kyuuba.

\subsection{The IPC System}
All communications between the programmes will be done using S-expressions in
regular text form. Spawned processes will be connected to the spawning programme
via regular stdio file descriptors. Stdin and stdout file descriptors have an
obvious use here; stderr will be connected to the monitor programme for error
logging.

This type of communication makes debugging and coordinator/server testing
extremely easy, and S-expressions are extremely easy to parse. There could also
be a binary encoding for the S-expressions to increase I/O throughput, although
this is unlikely to be required as S-expressions should be sufficiently fast to
parse.

We'll also develop a simple library for programmes to use this IPC system.

\section{Metadata and Configuration}
Metadata and configuration data will be stored in flat files, using
S-expressions. For performance or compatibility reasons, alternative encodings
may also be used by writing configuration backend modules.

\subsection{Service Model}
Instead of the immensely popularised event-based model, we'll stick with a
dependency-based model to coordinate launching of services. This is because,
while dependencies may seem like heavier-weight than than launching services on
events, it's easier to coordinate all of the services so that all the order
constraints fit together, while still being able to achieve proper
parallelisation for starting services.

\subsection{Dependencies}
The dependencies will be expressed as a big graph; this will be a digraph with
four types of nodes and two types of edges. The first node is a special root
node that will be used as a starting point whenever services need to start.
Apart from that, we'll need two types of nodes for each service, a start node
and an end node, and we'll need `action nodes'. These action nodes lie `between'
the start and end nodes, and plain edges connect the start nodes through the
action nodes with an end node.

To enable a service, all we need to do is find a path from the root node to the
end node of a service. Once a path is found, it is simply walked and all the
actions described in the action nodes along the path are taken. If an action
fails, the node is disconnected and a new path is searched; this allows for
fallback behaviour, where different implementations of a specific service exist,
for example daemons like fcron and vixie-cron, which could both provide a mutual
cron service, and if fcron fails to start, its action node is disconnected, and
searching a new path would find one that uses an action node that starts
vixie-cron.

Once an end-node for a service is reached, a special shortcut edge is added
between the start node and the end node of that service. When subsequently
starting other services that depend on that service, the shortcut edge will be
used instead of the plain edges so that there's no special handling required to
avoid taking actions described in any action nodes.

To disable a service again, for example when shutting down, all that needs to be
done is searching a path from an end node to the root node, and clearing the
shortcut edges whenever a start node is passed. Since the graph is directed,
this means there's a need for extra action nodes for stopping a service, which
conveniently fits the whole model, since different actions need to be taken to
disable a service.

\subsubsection{Advantages}
This whole graph model has the advantage of easily being enhanceable with
additional node or edge types for more complex layouts. The model here should be
able to handle the most common dependency types, such as
x-requires-y(,z(,u(,v(...)))) and x-should-be-started-if-y-fails. With minimal
adjustments and by introducing simple coordinator nodes and edges, it'll also be
able to express dependencies such as x-must-be-started-before-or-after-y and
x-should-always-be-started-if-installed or one could add things like join-nodes,
where all the currently active graph walking operations that would need to pass
through such a node would be required to join up.

Depending on the 'resolution' of the graph, that is, for example, whether one
considers each mountpoint to be a distinct service or only the whole 'mounting'
issue, this also allows for an elegant parallelisation of tasks such as mounting
filesystems, like in the example, filesystem checks, or bringing up different
network interfaces, which oftentimes incur a notable lag due to wlan
authentication or dhcp configuration.

Also, this graph adds additional transparency to users and developers, sich each
state of the dependency graph can be visualised with tools like graphviz, to
help in tracing seemingly strange waiting behaviour of the overall coordination
and parallelisation process.

\subsection{Configuration}
Configuration data, as well as dependency metadata and action descriptions are
all to be stored in S-expressions in the applicable configuration directory
(either /kyu/etc, /etc/kyu or /kyu). There's not that much configuration data
going to be needed for kyuuba itself, most of the data would be for tweaking
services and for defining which services to start.

Dependency metadata needed to construct the dependency graphs will be written in
the same way, since S-expressions should be well-suited for this task and being
able to use the same parser for both reduces the size of the code.

\section{OS Requirements}
Trying to cope with everything we encountered was one of the major causes of
grief in eINIT. To prevent this, we'll instead be setting a number of
requirements for a host OS for kyuuba, which should help us avoid a lot of the
issues we encountered with eINIT.

\subsection{Initramfs/Micro-bootstrap-partition}
The cleanest way to use kyuuba will be by means of a custom initramfs or a small
partition with only kyuuba on it.

This method of using kyuuba will have a few very distinct advantages, such as
being able to actually unmount the root-filesystem(s) before shutting down,
since the code could easily use the chroot-facility to get into any filesystems
we want to use. Currently it's not possible to actually unmount a
root-filesystem on Linux systems, it's only possible to remount such a
filesystem read-only, which should be considered less `clean' than actually
unmounting the filesystem.

Also, this approach has the nice advantage of being more appealing by
introducing fewer special cases to consider when used in combination with
chroot-jailed services, as well as adjusting the configuration for daemons like
udev to our needs.

This obviously relies on a system with the ability to use chroots, but a similar
effect should be achievable on any system with filesystem namespaces (such as
plan9).

Windows/ReactOS seem to be lacking in support for both chrooting and filesstem
namespaces, so this doesn't apply to those systems.

\subsection{FS Layout}
Kyuuba will need to rely on a few characteristics of the used filesystems. In
general, depending on the use of kyuuba, we could easily eliminate the need for
certain configuration files in /etc, such as fstab and mtab, however this proved
too hard to get accepted by the community in eINIT, so this is not going to be a
required adjustment for kyuuba to work.

\subsubsection{General Layout}
Ideally, kyuuba will be used on a system with a proper FHS layout, and kyuuba's
monitor will run directly off an initramfs or a small bootstrap partition, where
it will be responsible for mounting the root filesystem(s) of an arbitrary
number of (compatible) systems to boot.

\subsubsection{Kyuuba Files}
For FHS-compliant Linux systems, all kyuuba-related files are to be put under
/lib/kyu or / of a given partition. All configuration data is to be stored under
/etc/kyu. Binaries under /sbin will probably not be needed, since FHS systems
are supposed to be bootstrapped using an initramfs or a similar filesystem that
contains kyuuba's monitor, which is well capable of searching for servers and
coordinators in /lib/kyu as opposed to /sbin or /bin.

In a pure, kyuuba-only system, /sbin/init may be a non-initramfs kyuuba monitor.

The monitors should alternatively accept the inverse of this layout, that is
/kyu/lib and /kyu/etc, since for certain uses, such as jailing services, this is
less intrusive. With this layout, /kyu and /kyu/etc should both work the same.

*BSD variants should be able to use the same convention without breaking any
traditional FS designs. Windows ports should also be able to follow this layout,
although the latter layout could be seen as less intrusive.

Alternative FS layouts should use analoguous locations. For example for project
`Marco Polo', we'd be using the regular /os/arch/lib and /config/kyu directories
if it's running off its own dedicated partition, or /os/arch/lib/kyu if it
wasn't.

\section{Module Interface}
One of the goals is to have a highly modular design so we can get a 1.0 release
out as fast as possible while still being able to extend the functionality
afterwards. For this reason, we need a well designed interface to extend the
core functionality.

For flexibility, we want to implement all of the following three methods to
provide a proper environment to work in. Some of the 'loaders' here would be
modules themselves, so that for specific uses of kyuuba, useless loaders could
easily be omitted.

All three of these methods will provide the exact same interface on the
source-code level, which means the burden of adjusting things for the method to
use is put on the build system, not the addon-developer.

\subsection{Shared Objects}
An obvious way to provide this type of extensibility is to use shared objects,
or .dll files in Windows. This type of interface is handy, but it has some
disadvantages in terms of memory use, and in general the compiler for both the
.so/.dll file and the loading programme can't do all the optimisations it would
normally be able to do. This typically results in a larger combined binary size,
both on disk and in memory.

We tackled this issue in eINIT by making meta-modules to load, that is make a
collection of shared objects, link them all together as a single shared object
which then loads the built-in modules. This approach led to a significant
reduction in memory use, while keeping large parts of the flexibilty of the
concept.

\subsection{Statically Linked Modules}
Another fairly obvious approach is linking modules directly into the core. This
approach allows the C-compiler to do more optimisations than it ordinarily
could, resulting in smaller binaries and a lower memory-profile. On the other
hand, it requires the modules to be available in either source or object form at
compile/link-time, which decreases the flexibility of this type of module a bit.
Still, implementing this type of module has obvious advantages on
memory-constrained systems.

\subsection{Module Hubs}
Since the core design of kyuuba also includes spawnable servers and a strong IPC
subsystem, it would be easy to make a server that has an IPC link to a monitor
or a coordinator, and which offers the same interface for shared-object and
statically-linked modules.

This may seem like an odd way to extend core functionality, however this method
has two distinct advantages over the other two methods: for one thing, it
increases the overall stability of the system, since this way modules can be
loaded in different address spaces even though they will all be able to work
together closely, thanks to the IPC subsystem. This hardens the monitor and
coordinators against issues like segmentation faults and memory limitations.
Also, since servers are supposed to be able to suspend themselves, this allows
reclaiming of memory from modules that aren't being used very often, by grouping
those together and suspending the corresponding server.

\end{document}
